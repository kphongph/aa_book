\chapter{Steiner Tree}

\par{
ในบทนี้เราจะศึกษาคุณสมบัติของ Steiner Tree ปัญหาของการหา
Steiner Tree จากกราฟที่กำหนดให้ 
}

\par{
เพื่อให้เข้าใจในปัญหา Stenier tree เราจะเริ่มด้วยการศึกษาปัญหาที่มีลักษณะคล้ายกันก่อน
นั้นก็คือปัญหาการหาค่าน้อยที่สุดของ Spanning tree (Minimum spanning tree, MST)
}

\par{
กำหนดให้ $G(S, E)$ เป็นกราฟที่ไม่กำหนดทิศทางและมีการเชื่อมโยงกัน (Undirected and
connected graph) โดยมี $S$ และ $E$ เป็นเซตของจุดยอดและเซตของเส้นเชื่อมระหว่างจุดของ
กราฟ $G$ ตามลำดับ Spanning tree ของ G คือ กราฟ $G'(S, E')$ โดยที่ $E' \subseteq E$
และ $G'$ ยังคงเป็น กราฟที่มีการเชื่อมโยงกันเช่นเดิม
}

\par{
ปัญหา Steiner Tree หรือ ปัญหาการหาค่าที่น้อยที่สุดของ Steiner tree นั้นเป็นปัญหาที่มี
ลักษณะคล้ายคลึงกับ ปํญหาการหาค่าน้อยที่่สุดสำหรับ  Spanning tree 
}
